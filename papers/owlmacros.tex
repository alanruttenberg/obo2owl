%%DOS General comments
%DOS I think that the favored solution - macro expansion of simple axioms with existential quantifier or value - needs to be emphasized more strongly.  I feel it gets a bit buried in the other stuff.

\documentclass[11pt]{article}
%%%%%%%%%%%%%%%%%%%%%%%%%%%%%%%%%%%%%%%%%%%%%%%%%%%%%%%%%%%%%%%
%%                    ** mymacros.tex **                     %%
%%%%%%%%%%%%%%%%%%%%%%%%%%%%%%%%%%%%%%%%%%%%%%%%%%%%%%%%%%%%%%%


\typeout{** loading mymacros.tex **}

\def\h{\hbox}

\newcommand{\mathcom}[3]{ \newcommand{#1}[#2]{\mbox{$#3$}}}
\newcommand{\remathcom}[3]{ \renewcommand{#1}[#2]{\mbox{$#3$}}}

\newcommand{\mdef}[2]{ \newcommand{#1}{\mbox{$#2$}} }
 
\mathcom{\y}{0}{\ \vdash\ }               % yeilds 
\mathcom{\ys}{0}{\vdash_{\cal S}}         % yeilds sub S
\mathcom{\yt}{0}{\vdash_{\theta}}         % yeilds sub theta

\mathcom{\absurd}{0}{\mathbf{f}}                 % absurdity
\mathcom{\imp}{0}{\ \rightarrow\ }            % implication arrow
\mathcom{\rimp}{0}{\ \leftarrow\ }            % implication arrow

\mathcom{\con}{0}{\ \wedge\ }                 % conjunction
\mathcom{\dis}{0}{\ \vee\ }                   % disjunction
\mathcom{\n}{0}{\neg}                     % negation
\mathcom{\dimp}{0}{\ \leftrightarrow\ }       % mat equiv

\mathcom{\corresponds}{0}{\ \Lleftarrow\! \! \Rrightarrow\ }


%%\mathcom{\th}{0}{\theta}                  % theta


\mathcom{\A}{0}{\forall}                  % universal quantifier
\mathcom{\E}{0}{\exists}                  % existential quant

\mathcom{\Dec}{1}{{\cal D}(#1)}           % D(#1)

\mathcom{\elt}{0}{\in}

\mathcom{\tuple}{1}{\langle #1 \rangle}

% \mathcom{\equivdef}{0}{\equiv_{\mbox{\em \tiny def}}}
% \mathcom{\eqdef}{0}{=_{\mbox{\em \tiny def}}}
\def\equivdef{\mathrel{\ \equiv_{\mbox{\em \tiny
def}}\ }}

\def\eqdef{\mathrel{\ =_{\mbox{\em \tiny def}}\ }}


% \newtheorem{Rule}{Rule}
% \newtheorem{Lemma}{Lemma}
% \newtheorem{Theorem}{Theorem}


% Redeclaration of symbols for intuitionistic logic
% see Diller p.126
\def\ineg{\mathop\sim}
\def\iimp{\mathbin\Rightarrow}
\def\iiff{\mathbin\Leftrightarrow}
% new symbols from amssymb
%\def\iimp{\mathbin\rightarrowtail}
%\def\iiff{\mathbin\leftrightsquigarrow}

\def\icon{\mathbin\curlywedge}
\def\idis{\mathbin\curlyvee}

%Definitions for RCC inserts
\newcommand{\notch}{\Longrightarrow\kern-16pt{/}\ \ \ }
\newcommand{\ch}{\Longrightarrow}
\newcommand{\pr}[1]{\mbox{\sf #1}}

% RCC relations
\mathcom{\C}{0}{\pr{C}}
\remathcom{\P}{0}{\pr{P}}
\remathcom{\O}{0}{\pr{O}}
\mathcom{\PO}{0}{\pr{PO}}
\mathcom{\PP}{0}{\pr{PP}}
\mathcom{\NTPP}{0}{\pr{NTPP}}
\mathcom{\NTPPI}{0}{\pr{NTPP}^{-1}}
\mathcom{\TPP}{0}{\pr{TPP}}
\mathcom{\TPPI}{0}{\pr{TPP}^{-1}}
\mathcom{\TP}{0}{\pr{TP}}
\mathcom{\NTP}{0}{\pr{NTP}}
\mathcom{\NTPi}{0}{\pr{NTPi}}
\mathcom{\EC}{0}{\pr{EC}}
\mathcom{\DC}{0}{\pr{DC}}
\mathcom{\DR}{0}{\pr{DR}}
\mathcom{\EQ}{0}{\pr{EQ}}
\mathcom{\CG}{0}{\pr{CG}}

\mathcom{\Pin}{0}{\pr{Pi}} % \Pi is Greek letter
\mathcom{\PPi}{0}{\pr{PPi}}
\mathcom{\NTPPi}{0}{\pr{NTPPi}}

\mathcom{\OP}{0}{\pr{OP}}

\mathcom{\conv}{0}{\pr{conv}}
\mathcom{\rsum}{0}{\pr{sum}}
\mathcom{\rdiff}{0}{\pr{diff}}
\mathcom{\rcompl}{0}{\pr{compl}}
\mathcom{\rprod}{0}{\pr{prod}}
\mathcom{\cp}{0}{\pr{cp}}
\mathcom{\Us}{0}{\pr{Us}}
\mathcom{\us}{0}{\pr{u}}
\mathcom{\NULL}{0}{\pr{NULL}}
\mathcom{\rnull}{0}{\emptyset}
\mathcom{\CONV}{0}{\pr{CONV}}
\mathcom{\CON}{0}{\pr{CON}}

% Adaptations for new latex compatibility
\renewcommand{\Box}{\square}
\def\cal{\mathcal}

% meta language in definitions etc
\mathcom{\ml}{1}{\;\;\;\; \mbox{#1} \;\;\;\;}


%-- TOPLOG SYMBOLS

\mathcom{\Co}{0}{{\cal C}_{0} \ }
\mathcom{\CoX}{0}{{\cal C}_{0}}
\mathcom{\Cop}{0}{{\cal C}_{0}^+ \ }
\mathcom{\CopX}{0}{{\cal C}_{0}^+}
\mathcom{\mcop}{0}{\models_{C_o^+}}

\mathcom{\Lop}{0}{{\cal L}_{0}^+ \ }
\mathcom{\LopX}{0}{{\cal L}_{0}^+}
\def\mlo{\models_{{\mathcal  L}_0}}
\def\mlop{\models_{{\mathcal  L}_0^+}}

\mathcom{\Ci}{0}{{\cal C}_{1} \ }
\mathcom{\CiX}{0}{{\cal C}_{1}}

\mathcom{\Io}{0}{{\cal I}_{0} \ }
\mathcom{\IoX}{0}{{\cal I}_{0}}
\mathcom{\Iop}{0}{{\cal I}_{0}^+ \ }
\mathcom{\IopX}{0}{{\cal I}_{0}^+}

%%\mathcom{\inter}{1}{\langle #1 \rangle}
\mathcom{\inter}{1}{i( #1 )}

\newfont{\myssfont}{cmss12}

\mathcom{\Maps}{2}{\, _{#1}\!\!\!\rightleftharpoons^{#2} \ }

\mathcom{\CotoST}{0}{\Maps{C_0}{ST}}
\mathcom{\CtoST}{0}{\Maps{C\,}{ST}}
\mathcom{\IotoST}{0}{\Maps{I_0}{ST}}

\mathcom{\uni}{0}{\cal U \, }
\mathcom{\uniX}{0}{\cal U}
\mathcom{\U}{0}{\cal U}

\mathcom{\eu}{0}{\,=\,\uni}

\mathcom{\ol}{1}{\overline{#1}}

\mathcom{\Lsse}{0}{{\cal L}_{sse}}
\mathcom{\Lssei}{0}{{\cal L}_{ssei}}

\mathcom{\Luse}{0}{{\cal L}_{use}}
\mathcom{\Lusei}{0}{{\cal L}_{usei}}

\mathcom{\disdots}{0}{\dis\!\!\!\ldots\!\!\dis}
\mathcom{\condots}{0}{\con\ldots\con}





%% Miscelleneous


\def\HA#1#2{\pr{Holds-At}(#1,#2)}
\def\col{\!:\!}

% Enclose arguments in square double brackets to
% refer to semantic value of an expression
\def\semvalue#1{[\mkern-3mu[#1]\mkern-3mu]}

\def\forces{\mathop{\setbox0=\hbox{$\vdash$}\ \rule{0.04em}{1\ht0}\mkern1.5mu\box0}}

\mathcom{\hence}{0}{\Longrightarrow \hspace{2em}}

\def\sliderule{\centerline{\rule{4in}{1mm}}}

\def\Cbox{\mathop{\square\mkern-10mu
                   \mbox{\raise0.45ex\hbox{\tiny\sf C}}
                  \mkern2mu
                 }
         }

\def\letterbox#1{\mathop{\square\mkern-10mu
                   \mbox{\raise0.2ex\hbox{\small\sf #1}}
                  \mkern2mu
                 }
         }

%\def\tBox{\letterbox{\lower0.4ex\hbox{t}}}
\def\sBox{\letterbox{s}}
\def\tBox{\letterbox{t}}

\def\overstrike#1#2{\mathop{%
               \setbox1=\hbox{$#1$}%
               \setbox2=\hbox{$#2$}%
               \ifdim 1\wd1<1\wd2 % 
                    { \rlap{\mbox{\kern0.5\wd2\kern-0.5\wd1 \box1}} \box2 }
              \else {   \rlap{\mbox{$#1$}}
                        \rlap{\mbox{\kern0.5\wd1\kern-0.5\wd2 \box2}}
                        \phantom{#1} } 
              \fi }}

\def\rbox{\boxminus}
\def\rdia{\overstrike{\Diamond}{-}}
\def\cdia{\overstrike{\Diamond}{\rule{0.04em}{1.5ex}}}
\def\cbox{\inbox{\rule{0.04em}{1.6ex}}}
\def\cbox{\mathop{\setbox0=\hbox{$\square$}
                  \overstrike{\square}{\hbox{\rule{0.04em}{1\ht0}}}
             }}

\def\diaplus{\overstrike{\Diamond}{+}}




%% Make a modal box with a symbol centred inside
\def\inbox#1{\mathop{\setbox0=\hbox{$\square$}%
                      \setbox1=\hbox{#1}%
                      \rlap{\hbox{$\square$}}
                      \rlap{\mbox{\kern0.5\wd0\kern-0.5\wd1%
                                  \box1}}%
                      \phantom{\square}}}

\def\ibox{\mathop{\square\mkern-10mu
                   \mbox{\raise0.45ex\hbox{\tiny\em i}}
                  \mkern7mu
                 }
         }
\def\cdiamond{\mathop{\Diamond\mkern-12.7mu
                   \mbox{\raise0.4ex\hbox{\tiny\em c}}
                  \mkern7mu
                 }
         }


\def\s5box{\mathop{\square\mkern-10mu
                   \mbox{\raise0.45ex\hbox{\tiny 5}}
                  \mkern7mu
                 }
         }

\def\Fbox{\mathop{\square\mkern-11mu
                   \mbox{\raise0.45ex\hbox{\tiny \bf F}}
                  \mkern2mu
                 }
         }

\def\Box{\mathop\square}
\def\Diamond{\mathop\lozenge}

\def\bigI{\mathop{%
\hbox{%
%%
\def\thickness{1pt}%
\def\width{6pt}%
\def\height{0.6em}%
\def\gap{0.15em}%
%%
\dimen0=\thickness%
\divide\dimen0 by 2%
\dimen1=\width%
\divide\dimen1 by 2%
%%
\kern\gap%
\rlap{\vrule width\width height\thickness depth0pt}%
\rlap{\kern\dimen1 \kern-\dimen0%
\vrule width\thickness height\height depth0pt}%
\raise\height\hbox{\vrule width\width height\thickness depth0pt}%
\kern\gap}}}

\def\mconv{\mathop{% 
      \mbox{\raise0.35ex\hbox{$\scriptstyle\bigcirc$}}
                  }
          }
\def\mpconv{\circleddash}



% \def\putat#1#2#3{\setbox1=\hbox{#3}%
% \rule{0pt}{0pt}%
% \vspace*{#2}%
% \hspace*{#1}%
% \rule{0pt}{0pt}%
% \hbox to 0em{\rlap{\vtop to 0ex{\hbox to 0em{#3}}}}%
% \hspace*{-#1}\vspace*{-#2}}
% %\rule{0pt}{0pt}
% %\vspace*{-\ht1}\rule{0pt}{0pt}
% %\vspace*{-\ht1}

\def\putat#1#2#3{\rlap{\hbox{%
\kern#1% 
\vtop{\hbox{}%
\hbox{\vbox to #2{}} \hbox{#3}}}}}

% use putat within \hbox
% Objects will be located relative to baseline
% at start of the \hbox
% eg:
% \hbox{
% \putat{2in}{2in}{xxxxx}%
% \putat{2in}{2in}{00000}%
% \putat{5in}{3in}{Z}%
% \putat{2in}{2in}{0}%
% \putat{5in}{4in}{x}%
% }


%\long\def\putat#1#2#3{




\def\ri{\mathclose{\raise1ex\hbox{$\smile$}}}

\def\eqtag#1{\eqno ({\bf #1})}

\def\mc:#1{\hbox{${\mathcal{#1}}$}}
\def\mb#1{{\mathbf{#1}}}
 
\def\ifff{\qquad \hbox{if and only if} \qquad}

%%\def\d{\delta}

\def\QED{$\blacksquare$}

\def\epsrcgrant{GR/K65041}
\def\VUGgrant{GR/M56807}

\def\Boxx{\boxtimes}

%% Force a math symbol to go into scriptsize
\def\mscriptsize#1{\hbox{\scriptsize $#1$}}
\def\mlarge#1{\hbox{\large $#1$}}

\def\boxtheorem#1#2{~\\
\centerline{\fbox{
\parbox{5in}{
\centerline{\bf #1}
#2
}}}
\vspace{1ex}}

\def\proof#1#2{
\begin{quotation}
\noindent {\bf Proof of #1:}
#2
\QED
\end{quotation}
}

\def\proofpar{\\ \hspace*{1.5em}}


%%%% ENVIRONMENTS

%% specify my list parameters
%% These can be overridden by declarations within
%% a document.

\def\mytopsep{1ex}
\def\mypartopsep{0ex}
\def\myitemsep{0.3ex}
\def\myparsep{0ex}
\def\mylistparindent{2em}
\def\myleftmargin{4.5em}
\def\mylabelwidth{4.5em}
\def\mylabelsep{1em}
\def\myitemindent{0em}

\def\setmylistparams{%
             \setlength{\topsep}{\mytopsep}
             \setlength{\partopsep}{\mypartopsep}
             \setlength{\itemsep}{\myitemsep}
             \setlength{\parsep}{\myparsep}
             \setlength{\listparindent}{\mylistparindent}
             \setlength{\leftmargin}{\myleftmargin}
             \setlength{\labelwidth}{\mylabelwidth}
             \setlength{\labelsep}{\mylabelsep}
             \setlength{\itemindent}{\myitemindent}
        }

\def\clistbullet{$\bullet$}

\newcounter{tempvalue}
% A compact list making environment
\newenvironment{clist}
    { %\vspace{1ex}
      \begin{list}
            {\labelitemi}
            \setmylistparams
    }
    { \end{list}\addvspace{1ex} 
    }
% A compact list making environment
% with -- as the default item tag
\newenvironment{cdashlist}
    { %\vspace{1ex}
      \begin{list}
            {--}
            {\setlength{\topsep}{0.2ex}
             \setlength{\partopsep}{0ex}
             \setlength{\itemsep}{0.3ex}
             \setlength{\parsep}{0ex}
             \setlength{\listparindent}{2em}
            }
    }
    { \end{list}\addvspace{1ex} 
    }


% compact list with wide labels for item tags
\newenvironment{taglist}[1]
    { %\vspace{1ex}
      \begin{list}
            {$\bullet$}
            {\setlength{\topsep}{0.2ex}
             \setlength{\partopsep}{0ex}
             \setlength{\itemsep}{0.3ex}
             \setlength{\parsep}{0ex}
             \setlength{\listparindent}{2em}
             \setlength{\leftmargin}{#1}
             \setlength{\labelwidth}{#1}
             %\addtolength{\labelwidth}{-1em}
             \setlength{\labelsep}{0em}
            }
    }
    { \end{list}\addvspace{1ex} 
    }

\newcounter{cenumcount}
\newenvironment{cenum}
    { %\vspace{1ex}
      \begin{list}
            {\arabic{cenumcount}.}
            {\usecounter{cenumcount}
               \setmylistparams
            }
    }
    { \end{list}\addvspace{1ex} 
    }

\newcounter{romcount}
\newenvironment{romanenum}
          { 
            \begin{list}{\roman{romcount})~~}
                        {\usecounter{romcount}
                         \setmylistparams
                        }
          }
          { \end{list} }

\newcounter{alphcount}
\newenvironment{alphenum}
          { \begin{list}{\alph{alphcount})~~}
                        {\usecounter{alphcount}}}
          { \end{list} }


%% Labelled list environment
\newenvironment{llist}[1]{%
\newcounter{#1}
\begin{list}{({\bf #1\arabic{#1}})\hspace{1em}}{\usecounter{#1}}
}
{\end{list}}

%% Labelled list continued
%% (assumes counter already exists)
\newenvironment{llistcont}[1]{%
\setcounter{tempvalue}{\value{#1}}
\begin{list}{({\bf #1\arabic{#1}})\hspace{1em}}{%
\usecounter{#1}
\setcounter{#1}{\value{tempvalue}}
}
}
{\end{list}}

%%% compact versions of llist and llistcont
\newenvironment{cllist}[1]{%
\newcounter{#1}
\begin{list}{({\bf #1\arabic{#1}})}
             { \usecounter{#1}
               \setmylistparams
             }
}
{\end{list}}

%% Labelled list continued
%% (assumes counter already exists)
\newenvironment{cllistcont}[1]{%
\setcounter{tempvalue}{\value{#1}}
\begin{list}{({\bf #1\arabic{#1}})}{%
                 \usecounter{#1}
                 \setcounter{#1}{\value{tempvalue}}
                \setmylistparams
          }
}
{\end{list}}


\def\longdef{\long\def}

%% Define a labelled list type
%% #1 name of control sequence for the list type
%% #2 prefix for numbering
\def\llisttype#1#2{%
\newcounter{#2}
\expandafter\longdef\csname #1list\endcsname##1{%
\begin{cllistcont}{#2}
##1
\end{cllistcont}}
\expandafter\def\csname #1\endcsname##1{%
\begin{cllistcont}{#2}
\item
##1
\end{cllistcont}}
}

\def\litem#1#2{%
\begin{clist}
\item[({\bf #1})] 
#2
\end{clist}}

\newenvironment{chapabst}%
{\begin{quotation}\small\noindent }{\end{quotation}}

% Split line displayed formulae
% (not actually an environment)
\newcommand{\splitdismath}[4]{%
{\setlength{\arraycolsep}{0em}
\begin{eqnarray}
& #1 & #2 \nonumber \\
 &    & #3 
\label{#4}
\end{eqnarray}
}
}

% Displayed equation with tabbing

\newcommand{\tabmathn}[2]
{ \begin{equation}
\vbox{
\vspace{-2ex}
\begin{tabbing}
#1
\end{tabbing}
\vspace{-3.5ex}
}
\label{#2}
\end{equation}
}

\newcommand{\displaytab}[1]
{ \begin{displaymath}
\vbox{
\vspace{-2ex}
\begin{tabbing}
#1
\end{tabbing}
\vspace{-3.5ex}
}
\end{displaymath}
}

\def\myeq#1{$$\eqalign{#1}$$}


%%% Reinstate \eqalign macros removed by Lamport
\catcode`\@=11
\def\eqalign#1{\null \,\vcenter {\openup \jot \m@th 
\ialign {\strut \hfil $\displaystyle{##}$&$\displaystyle {{}##}$\hfil
 \crcr #1\crcr }}\,}

\def\eqalignno#1{\displ@y \tabskip \centering 
\halign to\displaywidth {\hfil $\@lign \displaystyle{##}$\tabskip \z@skip
 &$\@lign \displaystyle {{}##}$\hfil \tabskip \centering 
 &\llap {$\@lign ##$}\tabskip \z@skip \crcr 
 #1\crcr }}

\def\leqalignno#1{\displ@y \tabskip \centering
 \halign to\displaywidth {\hfil $\@lign \displaystyle {##}$\tabskip \z@skip
 &$\@lign \displaystyle {{}##}$\hfil \tabskip \centering
 &\kern -\displaywidth \rlap {$\@lign ##$}\tabskip \displaywidth \crcr 
 #1\crcr}}

%%% Define something like \bibliography but only records the
%%% bibfiles (does not input the bbl file)
\def\bibliographyfiles#1{%
  \if@filesw
    \immediate\write\@auxout{\string\bibdata{#1}}%
  \fi}

%%% Define a separate command to input the bbl file.
\def\bibliographytext{%
  \@input@{\jobname.bbl}}


%% Suppress printing of bibitems in thebibliography
%% This is for use with the `inlinebib' package,
%% which adds fullcitations where you give the \cite command.
%% Doesn't print note
%% Could add note by changing ##2 to ##2##3 in the \bibcite argument
%% But output not quite right.
\def\suppressendbib{%
\long\def\@bibitem##1 ##2 \note ##3 \short ##4 \end{\par\if@filesw
        {\def\protect####1{\string ####1\space}%
         \let\newblock\@empty
         \immediate\write\@auxout{\string
                \bibcite{##1}{\string\@bibcall{##1}{##2}{##4}}}}\fi
        }
\def\thebibliography##1{}
}


%% Change indent space allocated by \thebibliography
%% use before \bibliography or \bibliographytext 
\def\setbibindent#1{%
\let\oldthebibliography\thebibliography
\def\thebibliography##1{\oldthebibliography{#1}}
}



\catcode`\@=12

\long\def\skipover#1{}

\long\def\correction#1{\footnote{TO CORRECT: #1}}
\def\nocorrections{\long\def\correction##1{}}
\def\question#1{{\bf Question:} {\em #1} }
\long\def\todo#1{{\bf To Do:} {\em #1} }


\def\twiddle{$\sim$}

%% Macro for URLs
%% puts them in tt format and  breaks line after dot if needed.

\skipover{
\def\urlspace{-0.5em}
\def\url#1{{\tt \urlsub#1\end}}
%\def\urldot{. \hspace{-1em}\urlsub}
\def\urldot{. \linebreak[1]\hspace\urlspace\urlsub}
\def\urlcolon{: \linebreak[1]\hspace\urlspace\hspace\urlspace\urlsub}
\def\urlslash{/ \linebreak[1]\hspace\urlspace\urlsub}
\def\urltilde{\twiddle\urlsub}
\def\urlsub#1{\ifx#1\end \let\next=\relax
      \else \ifx#1.\let\next=\urldot
      \else \ifx#1:\let\next=\urlcolon
      \else \ifx#1/\let\next=\urlslash
      \else \ifx#1~\let\next=\urltilde
      \else#1\let\next=\urlsub\fi\fi\fi\fi\fi\next}
}

% Input a file in verbatim mode (eg a program file)
\def\verbinput#1{\expandafter\begin{verbatim}\input#1}
%NB: usage: \verbinput{filename}\end{verbatim}


\def\mb#1{{\mathbf{#1}}}
\def\mbf#1{{\mathbf{#1}}}


%%% MACROS for SLIDES


\def\heading#1{%
\centerline{\shadowbox{
\large \begin{tabular}{c} #1 \end{tabular}
           }  }
\vspace{1ex minus 1ex}
}

\def\bheading#1{%
\centerline{\shadowbox{
\large\bf \begin{tabular}{c} #1 \end{tabular}
           }  }
\vspace{1ex minus 1ex}
}

\def\printlandscape{\special{landscape}}    % Works with dvips.
\def\printportrait{\special{portrait}}

\def\leedslogo#1{\centerline{
\psfig{file=uol.eps,width=#1} %% May look wrong in xdvi
}}

\def\titleslide#1#2{
\begin{slide*}
~

\vfill
\heading{#1}

\vfill
\begin{center}
        {\large\bf Brandon Bennett} 

\vspace{3ex}
        Division of AI \\
        School of Computing \\
        University of Leeds\\ 
        Leeds LS2 9JT, England \\
        {\tt brandon@comp.leeds.ac.uk} \\
\end{center}

\vfill
\centerline{
\psfig{file=uol.eps,width=0.7in} %% May look wrong in xdvi
}
\vfill
\centerline{#2}
\vfill
\end{slide*}
}


\def\nlf{\\ \mbox{} \hfill}

%%\def\fixhyphens{\usepackage[english]{babel}}

%% FIX HYPHENATION
\lefthyphenmin=2
\righthyphenmin=3

%% special commands for spell checker
\def\sic#1{#1}
\def\sico!#1!{#1}
\def\sicname#1{#1}
\def\accept#1{}
\def\acceptname#1{}

%\newenvironment{nospell}{}{}
\def\spelloff{}
\def\spellon{}


%% Check whether a macro is defined
\def\ifundefined#1{\expandafter\ifx\csname#1\endcsname\relax}

%% Define \ifpdf
%% Used in graphic stuff to check if pdf is running
%% This is taken from ifpdf.sty
\newif\ifpdf
\ifx\pdfoutput\undefined
\else
  \ifx\pdfoutput\relax
  \else
    \ifcase\pdfoutput
    \else
      \pdftrue
    \fi
  \fi
\fi

\ifpdf
  \typeout{Running in PDF mode}
\else
  \typeout{Running in DVI mode}
\fi

%%%% CLEVER GRAPHIC INCLUSION

\def\pdfgraphictype{jpg} %% or perhaps pdf or png
\ifpdf   
  \def\graftype{\pdfgraphictype}  
\else
  \def\graftype{eps}
\fi


\typeout{* BB mymacros: Default graphic type set to: \graftype}

\ifpdf
   \def\optpdftex{pdftex}
   \def\optgraphic[#1]#2#3{\includegraphics[#1]{#3}} 
\else
   \def\pdfinfo#1{}
   \def\optpdftex{}
   \def\optgraphic[#1]#2#3{\includegraphics[#1]{#2}} 
\fi

\def\altgraphic[#1]#2#3#4{\optgraphic[#1]{#2.#3}{#2.#4}}

\def\figurepath{}
\def\graphswap[#1]#2{\optgraphic[#1]{\figurepath#2.eps}{\figurepath#2.\pdfgraphictype}}

%% \graphicifpdf includes graphic in pdf mode
%% otherwise just draws a box with the file name
%% I haven't really used this much.
\ifpdf
     \def\graphicifpdf[#1]#2{\includegraphics[#1]{#2}}
   \else
     \def\graphicifpdf[#1]#2{\fbox{\parbox{1in}{#1\\ #2}}}
\fi 

%\def\setaltgraphic#1#2{%
%   \def\altgraphic[##1]##2{\optgraphic[##1]{##2.#1}{##2.#1}}}

\def\graphic[#1]#2{\includegraphics[#1]{#2.\graftype}} 

\def\psfiggraphic{\renewcommand{\graphic}[2][]{\psfig{##1,file=##2.eps}}}

%%% Date functions

\def\dayth{\number\day\ifcase\day \or
st\or nd\or rd\or th\or th\or th\or th\or th\or th\or th\or 
th\or th\or th\or th\or th\or th\or th\or th\or th\or th\or
st\or nd\or rd\or th\or th\or th\or th\or th\or th\or th\or
st\fi}

\def\monthname{\ifcase\month \or
January\or February\or March\or April\or May\or June\or
July\or August\or September\or October\or November\or December\fi}


%%%% How to set up new font sizes
%% define a font which is Helvetica at 4.5 pt
\newfont{\boxfont}{hv at 4.5 pt}

% How to rename or alter the References section
% \renewcommand\refname{References\footnote{The   reference  list  given
%     here is somewhat incomplete due to limited preparation time. It is
%     intended  to  add several  more  recent  references  to the  final
%     version of this paper, relating the discussion to current works on
%     ontology      development.}}      


\typeout{** finished loading mymacros.tex **}

%%%%%%%%%%%%%%%%%%%%%%%%%%%%%%%%%%%%%%%%%%%%%%%%%%%%%%%%%%%%%%%
%%  END END END         (of mymacros.tex)      END END END   %%
%%%%%%%%%%%%%%%%%%%%%%%%%%%%%%%%%%%%%%%%%%%%%%%%%%%%%%%%%%%%%%%


\title{Macro expansion for OWL}

\author{Chris Mungall, Alan Ruttenberg, David Osumi-Sutherland, ...}

\def\PE{\pr{Physical Entity}}
\def\Pr{\pr{Process}}
\def\Sys{\pr{System}}
\def\Snap{\pr{Snapshot}}
\def\State{\pr{State}}
\def\Property{\pr{Property}}
\def\Tp{\pr{Timepoint}}
\def\pred{\pr{Predicate}}
\def\lacksPart{lacksPart}
\def\capableOf{capableOf}

\begin{document}

\maketitle

\begin{abstract}

Accurate representation of complex domains such as biomedicine demands
powerful and expressive ontology languages such as OWL. However, the
complex nested class expressions required for modeling can be a
hindrance to ontology authoring and adoption. These class expressions
can appear opaque to domain experts, and even users proficient in OWL
can benefit from some kind of syntactic sugar or ``short-cut''
strategy, especially when authoring large ontologies.

One solution is to have domain experts fill in simple templates (for
example, in Excel) and translate the results into more complex axioms,
but this has the disadvantage of being disconnected from full ontology
authoring and reasoning environment.

We present here a method of specifying shortcut properties directly in
OWL. These shortcut properties can be used in similar ways as object
properties within the OWL environment, with the resulting simple
axioms translated automatically to more complex axioms via macro
expansion. We describe some example scenarios where this is of use in
authoring existing bio-ontologies.

One of the main implications of this work is a way to simplify the
translation between OBO format and OWL, and the use of RDF
triple-stores with complex OWL ontologies.

\end{abstract}

\section{Introduction}

% TODO: 
%  number axioms /ref

Accurate representation of complex domains such the biosciences demand
powerful and expressive ontology languages such as OWL. However, the
very expressive power of OWL can be a hindrance to widespread adoption
and effective use - OWL axioms and class expression may not always
mirror the internal cognitive models of domain experts. 

In addition, the development of large ontologies often requires the
repeated application of the same templates for different combinations
of classes. Table \ref{tab:example-templates} shows some examples.

  \begin{table}
    \begin{tabular}{ | p{7cm} | p{7cm} | }
      \hline 
      \textbf{Template} & \textbf{Example} \\
      \hline

      hasPart some ('plasma membrane' and hasPart ?Y)

      &

      Class: 'alpha-beta T cell'
      EquivalentTo: 'T cell' and 
      hasPart some ('plasma membrane' and hasPart 'alpha-beta TCR complex')

      \\

      \hline

      bearerOf some (realizedBy only ?Y)

      &

      Class: 'immune cell'
      EquivalentTo: 'cell' and bearerOf some (realizedBy only 'immune system process')

      \\

      \hline

      (?X partOf some ?Y) (?Y hasPart some ?X)

      &

      Class: finger
      SubClassOf: partOf some hand

      Class: hand
      SubClassOf: hasPart some finger 
      \\

      \hline

      hasPart exactly 0 ?Y

      &

      Class: 'enucleate cell'
      EquivalentTo: cell and hasPart exactly 0 nucleus

      \\
      \hline
    \end{tabular}
    \caption{Example templates.}
    \label{tab:example-templates}
  \end{table}


To the seasoned ontologist accustomed to writing complex nested
expressions this may seem a trivial problem.  A common response to
this issue is to develop some concise ``syntactic sugar'' or
intermediate representation, and have this translated behind the
scenes to OWL. For example, on particular approach is to have domain
experts fill in an Excel table, with columns corresponding to
variables in the above, and to write a script to generate the OWL.

However, this approach can be unsatisfying for a number of reasons. A
scripting approach is ad-hoc and difficult to integrate with existing
OWL toolchains. In addition, the original intermediate representation
is lost on translation into OWL. Ideally the intermediate
representation would still be visible within generic OWL viewers and
development tools such as Protege.

%The
%ontology consumer (who may be typically be a domain scientist rather
%than a logician) also has to view the ontology at this ``low level''.

%However, for many users
%it may be preferable to work with the ontology at the template level.

Here we describe a macro expansion language that allows the
representation in a high-level intermediate form directly in OWL. A
generic tool can be used to translate back and forth between the
intermediate form and the expanded form. This allows users to view and
edit at either level, as appropriate.

%This representation has the advantage that the ``sweetened'' form can
%be directly represented in OWL-DL.

We also discuss the usage of a macro expansion language versus using
and extending existing OWL2 constructs such as property chains.

\section{Macro Expansion Language}

We propose a macro expansion language that can be represented directly
within OWL-DL. The macro expansions are represented as OWL Manchester
Syntax\cite{Horridge2006} with the addition of template
variables. Either annotation properties or object properties
(``shortcut'' relations) are annotated with macros, and these can be
used to expand individual axioms, either during the authoring process
or prior to reasoning. Table \ref{tab:macro-defining-props} shows the
two different annotation properties that can be used to define
shortcut relations, and table \ref{tab:macro-expansion} specifies how
expansion occurs.

  \begin{table}
    \begin{tabular}{ | p{3.5cm} | p{3.5cm} | p{5cm} | }
      \hline 
      \textbf{Domain} & \textbf{Annotation Property} & \textbf{Use} \\

      \hline
      Annotation Property & expandAssertionTo & Annotation assertion axioms connecting two classes \\

      \hline
      Object Property & expandExpressionTo & Class expressions \\

      \hline
    \end{tabular}
    \caption{Two annotation properties for describing shortcut relations.}
    \label{tab:macro-defining-props}
  \end{table}

  \begin{table}
    \begin{tabular}{ | p{4cm} | p{5cm} | p{5cm} | }
      \hline 
      \textbf{Annotation Property} & \textbf{Matches} & \textbf{Expansion} \\

      \hline
      expandAssertionTo & PropertyAssertion(?R,?X,?Y) & substitute(?R,?X,?Y) \\

      \hline
      expandExpressionTo & ?R some ?Y & substitute(?R,?Y) \\

      \hline
      expandExpressionTo & ?R value ?Y & substitute(?R,?Y) \\

      \hline
    \end{tabular}
    \caption{Macro expansion of shortcut relations.}
    \label{tab:macro-expansion}
  \end{table}

Table \ref{tab:examples} shows some examples. We now describe these in more detail.

\subsection{Macro Expansion of Annotation Properties}

We propose an annotation property expandAssertionTo that specifies how to expand
a single OWL PropertyAssertion (i.e. rdf triple) into a more complex
axiom by substituting two variables $?X$ and $?Y$ with the subject and
target of the assertion respectively.

The rule can be specified as:

\begin{verbatim}
PropertyAssertion(?R,?X,?Y) ==> substitute(?R,?X,?Y)
\end{verbatim}

For example:

\begin{verbatim}
AnnotationProperty: spatially_disconnected_from
Annotations: expandAssertionTo 
  "(part_of some ?X and part_of some ?Y) SubClassOf owl:Nothing"
\end{verbatim}

The following triple:

\begin{verbatim}
Class: brain
Annotations: spatially_disconnected_from 'spinal cord'
\end{verbatim}

would expand to a general class inclusion axiom:

\begin{verbatim}
(part_of some brain and part_of some 'spinal cord') SubClassOf owl:Nothing"
\end{verbatim}

TODO: HOW DO WE WRITE GCIs IN MANCHESTER?

%DOS GCIs?

However, the approach of expanding individual property assertions does
not work well when we want to use our high-level properties in
equivalence axioms.

Consider the following expansion rule:

\begin{verbatim}
ObjectProperty: has_plasma_membrane_part
Annotations: macro "Class ?X: SubClassOf has_part some 
   ('plasma membrane' and has_part some ?Y)"
\end{verbatim}

This works fine for:

\begin{verbatim}
Class: 'alpha-beta T cell'
Annotations: has_plasma_membrane_part some ('alpha-beta TCR complex')
\end{verbatim}

Which would expand to:

\begin{verbatim}
Class: 'alpha-beta T cell'
SubClassOf: has_part some 
   ('plasma membrane' and 
    has_part some 'alpha-beta TCR complex')
\end{verbatim}

However, because this expansion is only defined for property
assertions, it cannot be used in equivalence axioms. If the ontology
author wishes to write:

\begin{verbatim}
Class: 'alpha-beta T cell'
EquivalentTo: 'T cell' and 
   has_plasma_membrane_part some ('alpha-beta TCR complex')
\end{verbatim}

Then this requires a more complex means of treating expandAssertionTo
assertions. In addition, the above axiom is problematic as it is using
an annotation property where an object property should be used
[PUNNING?]

\subsection{Macro Expansion of Object Properties}

As an alternative that works for both SubClass and EquivalentTo
axioms, we propose a different expansion rule for a new annotation
property expandExpressionTo. Here, the macro describes how to expand a class
expression, not a property assertion, and only uses a single variable.

The macro object property should be used in existential restrictions,
and the translation rule is:

\begin{verbatim}
?R some ?Y ==> substitute(?R,?Y)
\end{verbatim}

For example:

\begin{verbatim}
ObjectProperty: has_plasma_membrane_part
Annotations: expandExpressionTo "has_part some ('plasma membrane' and has_part some ?Y)"
\end{verbatim}

And the following intermediate-form axiom:

\begin{verbatim}
Class: 'alpha-beta T cell'
EquivalentTo: 'T cell' and has_plasma_membrane_part some ('alpha-beta TCR complex')
\end{verbatim}

Would expand to:

\begin{verbatim}
Class: 'alpha-beta T cell'
EquivalentTo: 'T cell' and has_part some ('plasma membrane' and has_part some 'alpha-beta TCR complex')
\end{verbatim}

% Difference with what you're calling property assertion expansion quite subtle - esp as both use quantifiers. 

However, this is problematic if used in certain circumstances. Consider:

\begin{verbatim}
ObjectProperty: lacks_part
Annotations: expandExpressionTo "has_part exactly 0 ?Y"
\end{verbatim}

The author would write in the intermediate form:

\begin{verbatim}
Class: 'enucleate cell'
EquivalentTo: cell and lacks_part some nucleus
\end{verbatim}

Which would expand to:

\begin{verbatim}
Class: 'enucleate cell'
EquivalentTo: cell and has_part exactly 0 nucleus
\end{verbatim}

% Curious why you prefer cardinality approach rather than negation as Rob uses.

Note that the intermediate form would produce different inferences
from the expanded form. This is a major point against this scenario.

We can fix this by introducing an expansion rule for hasValue restrictions:

\begin{verbatim}
?R value ?Y ==> substitute(?R,?Y)
\end{verbatim}

Here we are treating the intermediate property as a relation between
an individual and a class.

Now we can write:

\begin{verbatim}
Class: 'enucleate cell'
EquivalentTo: cell and lacks_part value nucleus
\end{verbatim}

% I'd like a little more explanation of what this actually means.
% Also note - this may be legal OWL 2, but there are tooling issues.  P4 doesn't like this.  We should certainly test this against major reasoners.

This expands to the correct form. Whilst the intermediate form does
not produce all the same inferences as the intermediate form, the
intermediate form does not produce any incorrect inferences.

% So, the idea would be to use this value type expansions for these dangerous cases only?

We can also safely introduce an expansion rule for individuals


\begin{verbatim}
PropertyValue(?R,?X,?Y) ==> ?X Types: substitute(?R,?Y)
\end{verbatim}

We can write property assertions:

\begin{verbatim}
Individual: imaged-cell00001234
Properties: lacks_part nucleus
\end{verbatim}

\begin{verbatim}
Individual: imaged-cell00001234
Types: has_part exactly 0 nucleus
\end{verbatim}

\section{Extending Property Chains}

TODO - merge this with discussion.

OWL2 introduces additional more expressive constructs that reduces the
need for syntactic sugar or intermediate representations. One such
construct is property chains - we will first explore the potential for
using property chains to abstract away from complex repetitive axioms.

A biological ontology representing types of cells would likely include
classifications based on proteins expressed on the surface of the cell
(the \emph{plasma membrane}). The general template can be specified
as:

\begin{verbatim}
Class: ?X
EquivalentTo: 'T cell' and 
  has_part some ('plasma membrane' and has_part ?Y)
\end{verbatim}

For example, an $\alpha \beta$ T cell is a defined to be a \emph{T
  cell} that has an \emph{$\alpha\beta$ T cell receptor complex}
expressed on the plasma membrane:

\begin{verbatim}
Class: 'alpha-beta T cell'
EquivalentTo: 'T cell' and 
  has_part some ('plasma membrane' and has_part 'alpha-beta TCR complex')
\end{verbatim}

With OWL2 we can use property chains and self-restrictions to declare
a new property:

\begin{verbatim}
ObjectProperty: has_plasma_membrane_part 
SubPropertyChain: has_part o is_plasma_membrane o has_part

ObjectProperty: is_plasma_membrane

Class: 'plasma membrane'
SubClassOf: is_plasma_membrane Self
\end{verbatim}

%DOS Do you have this working in test ontologies?  Haven't managed to
%construct this is Protege. 4.0 and 4.02 don't appear to support Self.
%4.1 has other issues that prevent me testing.  I'd at least like to
%see this working with one of the major OWL reasoners.

This is somewhat complex for many domain experts who are not computer
scientists, but if there are a limited number of such properties then
the labor can be divided such that an OWL expert defines the
properties and the domain expert defines classes using these properties.

Now we can use this new property to define cells in terms of their
plasma membrane parts:

\begin{verbatim}
Class: 'alpha-beta T cell'
EquivalentTo: 'T cell' and has_plasma_membrane_part some ('alpha-beta TCR complex')
\end{verbatim}

Unfortunately, the semantics of this axiom is not exactly equivalent
to the previous fully expanded form. For example, if we ask for
subclasses of \emph{hasPart some plasma membrane} or \emph{hasPart
  some TCR receptor complex}\footnote{we assume transitivity of
  hasPart}, then only the first definition works. In fact, if we ask
for subclasses of \emph{hasPart some (plasma membrane and hasPart some
  abTCR complex)} then again only the first fully-expanded definition
works.

%DOS What about something that has_part some subtype of plasma membrane that has_part some subtype of alpha-beta TCR complex?  Would this be classified correctly?

 In order to have the
same semantics, the implication of the property chain would have to go
in both directions, but this is not permissible in OWL2.

%DOS Don't understand this point, and I suspect most readers wouldn't either.  For a start - isn't the property chain palindromic?
% Could you elaborate on this point?

It is important to point out that this may not be immediately obvious
to many OWL users. Many users may be using properties as if the
implication works in both directions.
% (for example, with the overlaps
%property, which can be inferred from a chain of hasPart and partOf).

% DOS But has_part is not symmetric, so why would users assume has_plasma_membrane_part to be?  Also, if we're pushing that individuals are used sparingly, then for the vast majority of uses, direction should come from quantifiers. Given that, isn't symmetry on this instance level (or lack of if) less of an issue?

One option here is to annotate the SubPropertyChain axiom:

\begin{verbatim}
SubPropertyChainOf:
  Annotations: isReversible "true"
  has_part o is_plasma_membrane o has_part
\end{verbatim}

This in itself does not change the semantics. However, some kind of
macro expansion tool could use this annotation to guide the expansion
of axioms that use the contracted form to the expanded form.

%DOS Perhaps inserting the expanded form here would make this point clearer?

The advantage here is that a subset of users (both ontology creators
and downstream users of ontologies) can view the ontology through a
simpler prism. The disadvantage is that it introduces complications -
if for some reasoner the axioms are not expanded it will change the
results of reasoning. Although in the above case it will not introduce
incorrect inferences, it will fail to produce some correct inferences.

The other main disadvantage here is that the intermediate
representation is not sufficiently expressive to abstract over other
useful common templates, such as:

\begin{verbatim}
bearer_of some (realized_by only ?Y)
\end{verbatim}

This kind of combination of an existential and universal restriction
over an intermediate anonymous class is common in many BFO-compliant
ontologies. For example:

\begin{verbatim}
Class: 'immune cell'
EquivalentTo: 'cell' and bearer_of some 
     (realized_by only 'immune system process')
\end{verbatim}

Immune cells are cells that bear some role or function that is
realized by an immune system process.  We could attempt to approximate
this using a property chain:

\begin{verbatim}
ObjectProperty: bearer_of_realized_by
SubPropertyChain: bearer_of o realized_by
\end{verbatim}

This could be used to write more compact axioms:

\begin{verbatim}
Class: 'immune cell'
EquivalentTo: 'cell' and 
   bearer_of_realized_by some 'immune system process'
\end{verbatim}

However, this does not have the same semantics as the original
expanded form, which combines an existential and universal
quantification. This means the ontology developer is forced to
explicitly write out the expanded form.

\section{Implementation}

We have implemented a tool called Annotation Property Expander (APEX)
that will apply these macros.

OK WE HAVENT YET BUT I CAN DO THIS QUICKLY


\section{Challenges}

Recursive expansion

% This would be useful.  I have a number of cases where I'm using one property that needs expansion to define another (in fact, I see you're using them below).
% However, it is not essential - The expansions for these complicated cases could just be written out in full.  
% Might be helpful to use these examples.

\section{Uses}


\section{Discussion}

\subsection{Do we need shortcut relations?}

Some ontology authors may have no objection to explicitly writing the
fully expanded forms of axioms. But this is not really practical for
the development of large ontologies by domain experts with
intermediate-level OWL skills. In these cases it is tremendously
beneficial to be able to use shortcut relations in axioms, especially
where these shortcut relations are used repeatedly. Ideally these
shortcut relations would be both usable within the standard
OWL-centric ontology authoring environment.

This also works well in terms of division of labor - a smaller number
of OWL experts can define a limited set of shortcut relations for a
particular domain, and a larger number of domain experts can apply
these.

\subsection{The hidden dangers of property chains}

Property chains are a powerful feature. However, the ontology author
must take care and not assume that a property restriction is
\emph{defining} the relation.

For example, a standard axiom in mereology\cite{Bittner} defines an
overlaps relation in terms of parthood:

\begin{verbatim}
overlaps(x,y) <-> exists z: hasPart(x,z), partOf(z,y)
\end{verbatim}

The overlaps relation is useful in bio-ontologies\cite{who}, so it can
be tempting to declare an object property for it, and provide a
property chain:

\begin{verbatim}
ObjectProperty: overlaps
SubPropertyChain: has_part o part_of
\end{verbatim}

But this axiom is weaker, and there is a danger if this property is
used in assertions thinking it has the same semantics. If an ontology
asserts that a overlaps some b, then the class hasPart some partOf b will
\emph{not} subsume a.

It may be useful to have ontology ``spell-checker'' detect when
properties with property chains are used in assertions, and ask the
author if they really intended to write out the fully expanded
form. As a matter of naming convention, it may be wise to call the
object property above something different, like ``infers-overlap''???

Self-restrictions add further power to property chains, by allowing
class membership to be 'checked' at points along the chain. But this
comes with the same health warnings as property chains, and an
additional disadvantage of adding opaque axioms to OWL classes, where
regular users will be looking, rather than burying the workings in
property annotations, where they probably won't. Presumably in future
there will be some syntactic sugar to hide this.

Overall, the temptation to define shortcut relations using property
chains should be avoided. Property chains should only be used when the
unidirectional implication is truly intended (and in these situations
they are of course very useful).

\subsection{Macro-expansion strategies}

Given that property chains are insufficient in themselves to support
the requirements for shortcut relations we have proposed some
strategies involving an explicit macro-expansion step.

We first explored the possibility of annotating property chain axioms
to indicate that the implication is intended to be
bi-directional. However, this is likely to cause confusion. We decided
it would be better to segregate shortcut relations completely from
other object properties.

We have explored two different types of shortcut relations. This first
type is modeled as an annotation property between two classes. This is
then expanded to a more complex axiom or collection of axioms -
possible a GCI axiom. These shortcut relations can not be used in
class expressions e.g. to define classes.

The second type can formally be seen as a relation between an
individual and a class. It expands to a more complex axiom involving
two classes. It can be used within class expressions, and can be used
to define classes. When used as a class expression, it can be used as
an existential or value restriction.

\subsection{Practical usage and implementation: ontology views}

The scheme we have proposed is relatively simple and can be
implemented easily.

However, to be maximally useful, the translator could be integrated
into ontology editing environments such as Protege4\cite{P4}. Expanded
axioms could be marked using axiom annotations. Users would have the
option of viewing the ontology at the intermediate level, the expanded
level, or both. Changes at one level would automatically
expand/contract to the other level, and feed in to incremental
reasoning.

However, even without this tight integration we believe that a
translation tool implementing the macro language specified here would
be useful.

There is a possible concern about reasoning over the intermediate
view, prior to expansion. However, our strategy here insulates us from
making incorrect inferences at the intermediate level. Users will
still have to be aware of the differences between the two levels to
know why they get more inferences in the expanded view.

% interestingly, we could write axioms specific to the intermediate
% view, effectively making it more expressive in some ways than the
% expanded view. But let's keep it simple and avoid this for now.

\subsection{Applications for OBO to OWL translation}

OBO-Edit is an ontology editing environment popular amongst
biologists\cite{Day-Richter2007}. The native model for this tool is
the OBO file format, which can be translated to a subset of
OWL\cite{Horrocks}\cite{tirmizi2009}. Whilst it is possible to write
nested class expressions in OBO format by using b-node style anonymous
classes, this is not well-supported and best avoided. In addition,
there is no standard way to write other OWL constructs such as
negation and universal restriction, which are becoming increasingly
necessary for bio-ontologies.

Many OBO-Edit authored ontologies have started informally adopting
shortcut relations with a view to translating these to expanded OWL
axioms further down the line. One approach is to do this as part of
the obo to owl translation, but we believe there are advantages to
doing the expansion in a separate layer. For one thing, it simplifies
the already complicated translation, which has some problems in the
existing translation. It also has the advantage of maintaining the
shortcut relations in the OWL environment.

Currently a piece of obo-format may look like this:

\begin{verbatim}
[Typedef]
id: ro:has_part
is_transitive: true
inverse_of: ro:part_of

[Typedef]
id: lacks_part
expandExpressionTo: "ro:has_part 0 ?Y" []

[Term]
id: cl:enucleate_cell
intersection_of: cl:cell
intersection_of: lacks_part go:nucleus
\end{verbatim}

Using a simple extension of the Horrocks proposal, this would translate to:

\begin{verbatim}
ObjectProperty: lacks_part
Annotations:
  expandExpressionTo "ro:has_part exactly 0 ?Y"

Class: cl:enucleate_cell
EquivalentTo: cl:cell and lacks_part value go:nucleus
\end{verbatim}

(PUNNING?)

Which would not lead to any incorrect inferences.

Applying the expansions rules we would get:

\begin{verbatim}
ObjectProperty: ro:has_part
Characteristics: Transitive

ObjectProperty: lacks_part
Annotations:
  expandExpressionTo "ro:has_part exactly 0 ?Y"

Class: cl:enucleate_cell
EquivalentTo: cl:cell and ro:has_part exactly 0 go:nucleus
\end{verbatim}

Which has the intended semantics.

\subsection{Applications for RDF and semantic web applications}

If an OWL ontology contains complex axioms involving nested class
expressions this can result in RDF triples that are difficult to work
with at the RDF level. This difficulty is ameliorated by languages
such as SPARQL-DL, but there is still problem for many triple stores
and RDF libraries that are not OWL-aware.

The shortcut relation strategy provided here could provide a means of
providing a ``triple-friendly'' view over complex OWL ontologies.

\subsection{N-ary relations}


\subsection{Automatically extracting shortcut relations}


\section{Conclusions}

\section{References}

\bibliographystyle{plain}
\bibliography{owlmacros}

\end{document}

